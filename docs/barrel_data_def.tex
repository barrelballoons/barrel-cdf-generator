\documentclass{article}
\usepackage[utf8]{inputenc}

\title{BARREL Data Definitions}
\author{Warren Rexroad}
\date{April 2013}

\usepackage{float}

\begin{document}

\maketitle

\section{Introduction}
This document will outline the different types of data products by BARREL. There are four types of data files that can be obtained: Raw data packets, Level 0, Level 1, and Level 2. Raw data packets are files that have a .pkt suffix and contain flight data exactly as they are received. Level 0 files are still in the raw data frame format but have been compiled into day-long files and had any bad frames removed. These files have a .tlm suffix.

Level 1 files are the first level of CDF files. they contain the exact integer numbers for each data type that is transmitted by the payload. That means that for digital values such as GPS\_TIME and X-Ray counts, the actual values are transmitted. However, with analog values such as housekeeping data, the data must be scaled before use. This means that much of the L1 data are not in physical units. Level 2 files are the same data from the L1 files, but with some additional processing. All analog values have been scaled, all accumulated values are converted to have /second units, spectral values are rebinned, and magnetometer values are gain corrected.

\section{File Name Conventions}
BARREL data file names all lower case and vary slightly depending if the file type is Raw Data File, Level 0, or Level 1/2 file all have slightly different file name formats:

Raw Data Files: bar\_YYMMDD\_HHMMSS\_PP\_GG.pkt

L0: barCLL\_PP\_S\_LV\_YYYYMMDD\_vVV.tlm

L1/2: barCLL\_PP\_S\_LV\_TTTT\_YYYYMMDD\_vVV.cdf

\begin{center}
\line(1,0){250}
\end{center}

YYMMDD / YYYYMMDD = Date

HHMMSS = Time

GG = Ground Station (SC - Santa Cruz, DC - Dartmouth College, SA - SANAE, HB - Halley Bay)

C = Campaign number (0 for non-campaign data, 1 for 2012/2013, 2 for 2013/2014)

PP = Payload ID

S = Launch Site (0 for non-campaign data, 1 for SANAE, 2 for Halley Bay)

LV =  Level (l0, l1, or l2)

LL = Launch order code (00 for non-campaign data)

VV = Version Number

TTTT = Data Type (sspc, mspc, fspc, gps-, pps-, magn, hkgn)

\subsection{Data Types}
The six data types available are the follow:

SSPC, MSPC, FSPC = Slow, Medium, or Fast Spectra. These are the spectral products returned by scintillator. SSPC returns 256 samples every 32 seconds, MSPC is 48 samples every 4 seconds, and FSPC is 20Hz data.

GPS- = GPS coordinate and time (ms of week in GPS Time) data. 

PPS = Pulse per second. This data type tells us how many milliseconds into the frame we receive the timing pulse from the GPS satellites.

MAGN = 3-axis, 4Hz Magnetometer data.

HKPG = Housekeeping data. These data have 40 multiplexed values (one returned each second) in each record. Voltages, currents, temperatures, and other diagnostics are kept here.

\section{Raw Telemetry and Level 0 Files}
Data are transmitted from the payloads to the ground station via Iridium modems. When a call between the two modems is connected a raw telemetry file is created. As data are received at the ground station, they are written, byte-by-byte to the data file with no integrity checks. Data are written to the file until the modem link fails and the call is dropped. When the call resumes, a new file is created.

These files are stored in a network accessible location and are downloaded daily by the CDF Generator. The data frame format for these files is described in the ''BARREL Telemetry Interface Control Document'' located at http://www.dartmouth.edu/\~{}barrel/documents.html.

After the CDF Generator has downloaded the .pkt files, its first output is a L0 file. This file is produced by checking the integrity of each frame in the raw data files and saving the good frames to the L0 file and rejecting the corrupt ones. Frames are flagged as being corrupt if they have a bad checksum, are too long or too short. 

During this process all of the .pkt files from a single date directory are processed. Unfortunately, because the call durations are arbitrary, it is possible for some of the previous or following day's data to be collected and saved to the wrong L0 file. This will be corrected in future versions of the CDF Generator.


\section{Level 1 and 2 Files}
The CDF Generator uses the daily L0 data to create L1 and L2 files. Level 1 and 2 data each split into 6 files based on data type: GPS, Magnetometer, Light Curves, Medium Spectra, Slow Spectra, and Housekeeping Data.

Before the L1 and L2 files are created, a time stamp is calculated for each frame. This is calculated by creating a linear model to convert frame number to time stamp based on groups of up to 2000 frames. This time stamp is recorded as the Epoch variable in each file. Epoch is stored as a TT2000 variable which is an integer that represents the number of nanoseconds since J2000.

In addition to the Epoch variable, each file contains a variable called FrameGroup. This variable contains the frame number provided by the payload for the data in each record. In the case of records that contain multiple frames of data, FrameGroup corresponds to the first frame in the group.

\subsection{GPS}
The GPS file contains mod4 data from the GPS antenna (Latitude, Longitude, Altitude and GPS Time). 

The differences between the L1 and L2 versions of the GPS file are shown in Tables 1 and 2.

\begin{table}[H]
\caption{L1 GPS Variables}
\begin{tabular}{|c|c|c|}
\hline
Variable Name&Variable Type&Description\\ \hline
GPS\_Lat&INT4&Latitude in $2^{31}$ semicircle\\
GPS\_Lon&INT8&Longitude in $2^{31}$ semicircle\\
GPS\_Alt&INT8&Altitude in mm above sea level\\
ms\_of\_week&INT8 & Milliseconds of week starting 0000 UTC Monday morning.\\
\hline
\end{tabular}
\end{table}

\begin{table}[H]
\caption{L2 GPS Variables}
\begin{tabular}{|c|c|c|}
\hline
Variable Name&Variable Type&Description\\ \hline
GPS\_Lat&INT4&Latitude in degrees\\
GPS\_Lon&FLOAT&Longitude in degrees\\
GPS\_Alt&FLOAT&Altitude in km above sea level\\
ms\_of\_week&INT8 & Milliseconds of week starting 0000 UTC Monday morning.\\
\hline
\end{tabular}
\end{table}

\subsection{PPS}
The PPS file contains 1Hz data. It holds both the PPS variable and the DPU version number that is transmitted in each frame. In both L1 and L2 the PPS variable is an INT4 that represents the number of milliseconds in to the frame that the GPS PPS signal was received. The DPU Version variable is INT2 in both L1 and L2 and has a valid range from 0-31.

\subsection{Rate Counters}
High Level, Low Level, and Peak Detector are counted on the analog board. Low Level and Peak Detector are for circuit diagnostics. Low Level counts excursions above a baseline and includes rejected events. Peak Detector counts peaks detected on the ADC board. For low count rate, low-noise environment, and at room temperature: Low Level = Peak Detect + High Level and Peak Detector = Interrupt.\\\\
Interrupt counts analyzed (ADC) x-rays as accepted by the DPU board.\\\\
The only difference between L1 and L2 data is that in L1 the units are counts/4seconds and in L2 the units are counts/second.

\subsection{Spectral Data}
Spectral data are collected by a 4096 bin NaI scintillator with a nominal bin to energy conversion of 2.4keV/bin. These data are collected and stored using three timing schemes: Fast, Medium, and Slow spectra.

\subsubsection{FSPC}
FSPC is the four-channel, fast spectra data which is recorded at 20Hz. Each frame is split into 20 CDF records, each record containing a variable for each channel.\\\\
The differences between L1 and L2 FSPC files are outlined in Tables 3 and 4.

\begin{table}[H]
\caption{L1 FSPC Variables}
\begin{tabular}{|c|c|c|c|}
\hline
Variable Name&Variable Type & Units & Description\\ \hline
LC1&INT4&cnts/.05sec&Bin Range:0-75, Energy Range:0-180keV\\
LC2&INT4&cnts/.05sec&Bin Range:76-230, Energy Range:182.4-552keV\\
LC3&INT4&cnts/.05sec&Bin Range:231-350, Energy Range:554.4-840keV\\
LC4&INT4&cnts/.05sec&Bin Range:351-620, Energy Range:842.4-1488keV\\
\hline
\end{tabular}
\end{table}

\begin{table}[H]
\caption{L2 FSPC Variables}
\begin{tabular}{|c|c|c|c|}
\hline
Variable Name&Variable Type&Units&Description\\ \hline
LC1&DOUBLE&cnts/sec&Counts are rebinned based on scintilator\\
~&~&~&and DPU temperature, and 511 line location \\
LC2&DOUBLE&cnts/sec &  ~\\ 
LC3&DOUBLE&cnts/sec & ~\\ 
LC4&DOUBLE&cnts/sec & ~\\
LC1\_ERROR&DOUBLE&cnts/keV/sec&Square root of counts in each channel\\ 
LC2\_ERROR&DOUBLE&cnts/keV/sec&~\\ 
LC3\_ERROR&DOUBLE&cnts/keV/sec&~\\ 
LC4\_ERROR&DOUBLE&cnts/keV/sec&~\\
\hline
\end{tabular}
\end{table}

\subsubsection{MSPC}
MSPC is 48 channel, medium spectra data which is an accumulation of counts over 4 seconds. Each record contains all 48 channels and is comprised of 4 frames. Unlike the FSPC spectra, all of the channels in the record are stored in an array who's indices are the channel number. A description of the nominal energy binning scheme can be found  in the ''BARREL Telemetry Interface Control Document'' located at http://www.dartmouth.edu/\~{}barrel/documents.html\\\\
The differences between L1 and L2 MSPC files are outlined in Tables 5 and 6.

\begin{table}[H]
\caption{L1 MSPC Variables}
\begin{tabular}{|c|c|c|c|}
\hline
Variable Name&Variable Type&Units&Description\\ \hline
MSPC&INT4[48]&cnts/4sec&Four second accumulation of 48 channel spectral data.\\
~&~&~&Nominal energy bin scheme.\\ 
ch&UINT[48]&NA&Values 1-48 used by the CDF for\\
~&~&~&labeling the MSPC array.\\
\hline
\end{tabular}
\end{table}

\begin{table}[H]
\caption{L2 MSPC Variables}
\begin{tabular}{|c|c|c|c|}
\hline
Variable Name&Variable Type&Units&Description\\ \hline
MSPC&DOUBLE[48]&cnts/keV/sec&L1 counts rebinned according to DPU\\
~&~&~&and scintillator temp and 511 line.\\ 
MSPC\_ERROR&DOUBLE[48]&cnts/keV/sec&Square root of binned counts.\\ 
ch&UINT[48]&NA&Values 1-48 used by the CDF for\\
~&~&~&labeling the MSPC array.\\
\hline
\end{tabular}
\end{table}

\subsubsection{SSPC}
SSPC is very similar to MSPC with the main difference that counts are accumulated over 32 seconds and split into 256 channels. Again, a description of the nominal energy binning scheme can be found  in the ''BARREL Telemetry Interface Control Document'' located at http://www.dartmouth.edu/\~{}barrel/documents.html\\\\
The differences between L1 and L2 SSPC files are outlined in Tables 7 and 8.

\begin{table}[H]
\caption{L1 SSPC Variables}
\begin{tabular}{|c|c|c|c|}
\hline
Variable Name&Variable Type&Units&Description\\ \hline
SSPC&INT4[256]&cnts/32sec&32 second accumulation of 256 channel spectral data.\\
~&~&~&Nominal energy bin scheme.\\ 
ch&UINT[256]&NA&Values 1-256 used by the CDF for\\
~&~&~&labeling the MSPC array.\\
\hline
\end{tabular}
\end{table}

\begin{table}[H]
\caption{L2 SSPC Variables}
\begin{tabular}{|c|c|c|c|}
\hline
Variable Name&Variable Type&Units&Description\\ \hline
SSPC&DOUBLE[256]&cnts/keV/sec&L1 counts rebinned acording to DPU\\
~&~&~&and scintillator temp and 511 line.\\ 
SSPC\_ERROR&DOUBLE[256]&cnts/keV/sec&Square root of binned counts.\\ 
ch&UINT[256]&NA&Values 1-256 used by\\
~&~&~&the CDF for labeling the SSPC array.\\
\hline
\end{tabular}
\end{table}

\subsection{Magnetometer}
The analog magnetometer data is encoded by a stand alone ADC. The data are collected from the X, Y, and Z axes at 4Hz and are transmitted in each frame. Each frame is split into 4 records. The digital word transmitted by the payload can be decoded with the following formula: $B_{analog} = \frac{B_{digital} - 8388608.0}{83886.070}$.\\\\
Gain correction is done based on temperature calibrations done preflight.\\\\
Tables 9 and 10 give info on L1 and L2 magnetometer files

\begin{table}[H]
\caption{L1 Magnetometer Variables}
\begin{tabular}{|c|c|c|c|}
\hline
Variable Name&Variable Type&Units&Description\\ \hline
MAG\_X&INT8&uT&Digital word from magnetometer ADC X axis.\\ 
MAG\_Y&INT8&uT&Digital word from magnetometer ADC Y axis.\\ 
MAG\_Z&INT8&uT&Digital word from magnetometer ADC Z axis.\\ 
\hline
\end{tabular}
\end{table}

\begin{table}[H]
\caption{L2 Magnetometer Variables}
\begin{tabular}{|c|c|c|c|}
\hline
Variable Name&Variable Type&Units&Description\\ \hline
MAG\_X&DOUBLE&uT&Nominal conversion of magnetometer X axis.\\ 
MAG\_Y&DOUBLE&uT&Nominal conversion of magnetometer Y axis.\\ 
MAG\_Z&DOUBLE&uT&Nominal conversion of magnetometer Z axis.\\ 
Total&DOUBLE&uT&Magnitude of B under nominal conversion.\\ 
MAG\_X\_ADJ&DOUBLE&uT&Gain corrected value of magnetometer X axis.\\ 
MAG\_Y\_ADJ&DOUBLE&uT&Gain corrected value of magnetometer Y axis.\\ 
MAG\_Z\_ADJ&DOUBLE&uT&Gain corrected value of magnetometer Z axis.\\ 
Total\_ADJ&DOUBLE&uT&Gain corrected magnitude of B.\\ 
\hline
\end{tabular}
\end{table}

\subsection{Housekeeping}
Housekeeping data are transmitted as digital words calculated by an ADC and multiplexed as mod40. These values are saved to the L1 files while the scaled (physical units) values are saved to the L2 files. The ''BARREL Housekeeping Assignments'' document gives the conversion factors for scaling the digital data and the ''BARREL Telemetry Interface Control Document'' lists the order in which the housekeeping data are transmitted.\\\\
Table 11 lists the variable information for L1 and L2 housekeeping files.\\\\

\begin{table}[H]
\caption{Housekeeping Variables}
\begin{tabular}{|c|c|c|c|p{4cm}|}
\hline
Variable Name&L1 Variable Type&L2 Variable Type&Units&Description\\ \hline
T0\_Scint&INT8&DOUBLE&$^\circ C$&Scintillator Temp\\
T1\_Mag&INT8&DOUBLE&$^\circ C$&Magnetometer Temp\\
T2\_ChargeCont&INT8&DOUBLE&$^\circ C$&Charge Controller Temp\\
T3\_Battery&INT8&DOUBLE&$^\circ C$&Battery Temp\\
T4\_PowerConv&INT8&DOUBLE&$^\circ C$&Power Converter Temp\\
T5\_DPU&INT8&DOUBLE&$^\circ C$&Data Processing Unit Temp\\
T6\_Modem&INT8&DOUBLE&$^\circ C$&Modem Temp\\
T7\_Structure&INT8&DOUBLE&$^\circ C$&Payload Structure Temp\\
T8\_Solar1&INT8&DOUBLE&$^\circ C$&Solar Panel 1 Temp\\
T9\_Solar2&INT8&DOUBLE&$^\circ C$&Solar Panel 2 Temp\\
T10\_Solar3&INT8&DOUBLE&$^\circ C$&Solar Panel 3 Temp\\
T11\_Solar4&INT8&DOUBLE&$^\circ C$&Solar Panel 4 Temp\\
T12\_TermTemp&INT8&DOUBLE&$^\circ C$&Terminate Temp\\
T13\_TermBatt&INT8&DOUBLE&$^\circ C$&Terminate Battery\\
T14\_TermCap&INT8&DOUBLE&$^\circ C$&Terminate Capacitor\\
T15\_CCStat&INT8&DOUBLE&$^\circ C$&Charge Controller Status\\
V0\_VoltAtLoad&INT8&DOUBLE&$V$&Voltage at Load\\
V1\_Battery&INT8&DOUBLE&$V$&Battery Voltage\\
V2\_Solar1&INT8&DOUBLE&$V$&Solar Panel 1  Voltage\\
V3\_POS\_DPU&INT8&DOUBLE&$V$&Data Processing Unit Positive Voltage\\
V4\_POS\_XRayDet&INT8&DOUBLE&$V$&X-ray detector Positive Voltage\\
V5\_Modem&INT8&DOUBLE&$V$&Modem Voltage\\
V6\_NEG\_XRayDet&INT8&DOUBLE&$V$&X-ray detector Negative Voltage\\
V7\_NEG\_DPU&INT8&DOUBLE&$V$&Data Processing Unit Negative Voltage\\
V8\_Mag&INT8&DOUBLE&$V$&Magnetometer Voltage\\
V9\_Solar2&INT8&DOUBLE&$V$&Solar Panel 2  Voltage\\
V10\_Solar3&INT8&DOUBLE&$V$&Solar Panel 3  Voltage\\
V11\_Solar4&INT8&DOUBLE&$V$&Solar Panel 4  Voltage\\
I0\_TotalLoad&INT8&DOUBLE&$mA$&Total Current at Load\\
I1\_TotalSolar&INT8&DOUBLE&$mA$&Total Solar Current\\
I2\_Solar1&INT8&DOUBLE&$mA$&Solar Panel 1 Current\\
I3\_POS\_DPU&INT8&DOUBLE&$mA$&Data Processing Unit Positive Current\\
I4\_POS\_XRayDet&INT8&DOUBLE&$mA$&X-ray Detector Positive Current\\
I5\_Modem&INT8&DOUBLE&$mA$&Modem Current\\
I6\_NEG\_XRayDet&INT8&DOUBLE&$mA$&X-ray Detector Negative Current\\
I7\_NEG\_DPU&INT8&DOUBLE&$mA$&Data Processing Unit Negative Current\\
numOfSats&INT2&INT2&$NA$&Number of GPS satellites in view.\\
timeOffset&INT2&INT2&$NA$&Number of leap seconds.\\
termStatus&INT2&INT2&$NA$&Terminate Status\\
cmdCounter&INT4&INT4&$NA$&Command Counter\\
modemCounter&INT2&INT2&$NA$&Modem Reset Counter\\
dcdCounter&INT2&INT2&$NA$&Number of times DCD has been de-asserted.\\
weeks&INT4&INT4&$NA$&Number of weeks since 6 Jan 1980.\\
\hline
\end{tabular}
\end{table}

\end{document}